\chapter{Introduction}
\section{Computational Sciences}
This work is built on two dissimilar research topics but in overlapping academic domains - Computational Biophysics and Analytical Chemistry. In case of Biophysics, the research work was solely based on algorithm development adapted to modern computational techniques in computer science. Specifically applied to simulate spectra for Electron Paramagnetic Resonance(Electron Spin Resonance) of nitroxide spin labels. In case of Analytical Chemistry, we performed pioneering study on application of Artificial Neural Networks compared to other advanced Machine Learning techniques for classification Alzheimer dementia on different progression stages(moderate and mild) based on Raman spectra of blood or cerebral spinal fluid.
\subsection{Electron Paramagnetic Resonance} 
Electron Paramagnetic Resonance(EPR) is a sub-branch of Nuclear Magnetic Resonance(NMR). In both NMR and EPR the interactions between electromagnetic radiation and magnetic moments are being treated. In EPR the magnetic moments arise from unpaired electrons rather then nuclei like in NMR. This spectroscopic method is widely used for investigating paramagnetic species ranging from detecting genital statistics disorders in women~\cite{2014arXiv1405.1230S} to measuring errors in single qubit rotations~\cite{2016arXiv161101110L}. Our group has a specific interest in simulating Nitroxide spin-labeled species in various motion regimes which are used as an activation site for EPR silent macromolecules providing the insights of its complex dynamics. \\ In simulating magnetic resonance spectra there are a few distinct algorithms exist which all have been developed during the mid 60's of the 20th century. Evolutionary the less computationally expensive approach have survived which is an eigenfunction expansion of density matrix~\cite{freed}. A multi‐trajectory estimation of	the spectral correlation function approach was introduced by~\cite{pedersen} and well used in our research group to model ESR Line Shape~\cite{mat_thesis}. An alternative approach~\cite{blume} that will be worked out in this thesis project is based on a master equation discretization and shares common features with modern molecular dynamics methods~\cite{sezer}. We will develop master equation discretization approach based on Blume work and describe major computational pitfalls and cost associated with it. 
\\
We will start with preparing reader for the theoretical minimum to understand the performed work. In Chapter 2 we will cover basic theory of Magnetic Resonance Spectroscopy from clasical  and statistical to quantum mechanical concepts. We will start from straight forward walk-through on phenomenological Bloch Equations describing the effect of nuclear magnetization and different relaxation mechanisms and continue with description of magnetization using density matrix approach and introducing Liouville operator.  
We will continue the discussion with the electronic $g$-tensor and hyperfine $A$-tensor anysotropies introducing their generalized forms that will be used to construct orientation dependent Hamiltonian from body to space-fixed coordinate frames where the applied magnetic field and spin operators are defined. We will also discuss distinct approach on parameterizing rotations via Euler angles or Quaternions. At the end of Chapter 2 we will describe Kubo-Anderson model\cite{kubo}\cite{anderson} of motional narrowing that was generalized by Blume which will be a fundamental for this work. In Chapter 3 we will describe developed algorithm and present the results. We will also compare number of discrete steps needed in order to completely model all spectral features.       
\subsection{Artificial Neural Network for Analytical Chemistry}
